\setcounter{chapter}{2}
\chapter{Polynomial Functions}
	\section{Exploring Polynomial Functions}
		\subsection{Polynomial Functions}
			A polynomial function $y=f(x)$ is defined by:
			\[f(x)=a_nx^n + a_{n-1}x^{n-1} + \dots + a_2x^2 + a_1x+a_0\]

			\begin{itemize}
				\item $a_n$, $a_{n-1}$, \dots , $a_1$, $a_0$ are real  numbers, called the coefficients of the polynomial function
				\item $a_n$ is called the leading coefficient
				\item $a_nx^n$ is called the leading term
				\item $a_0$ is called the constant term
				\item $n$ is a non-negative integer that gives the degree of the polynomial function
				\item The degree of the polynomial function is the largest exponent of $x$.
			\end{itemize}
		\subsection{Order}
			The terms of a polynomial function can be written in any order because addition is commutative.
		\subsection{Specific Polynomials}
			For $n=0$, $f(x)=a_0$ is called a constant function.

			For $n=1$, $f(x)=a_1x+a_0$ is called a linear function.

			For $n=2$, $f(x)=a_2x^2+a_1x+a_0$ is called a quadratic function.

			For $n=3$, $f(x)=a_3x^3+a_2x^2+a_1x+a_0$ is called a cubic function.

			For $n=4$, $f(x)=a_4x^4+a_3x^3+a_2x^2+a_1x+a_0$ is called a quartic function.

			For $n=5$, $f(x)=a_5x^5+a_4x^4+a_3x^3+a_2x^2+a_1x+a_0$ is called a quintic function.
		\subsection{Operations with Polynomial Functions}
			All the four operations ($+$, $-$, $\times$, and $\div$) are defined for polynomial functions.
		\subsection{y-intercept}
			The y-intercept of a polynomial function is equal to the constant term.
			\[f(0) = a_0\]
		\subsection{Finite Differences}
			The $n$th finite differences of a polynomial function of degree $n$ are constant. This constant $c$ is related to $a_n$ and $n$ by:
			\[c=n!a_n\]

			Given the $n$th difference, and $n$, you can solve for the leading coefficient of the polynomial:
			\[a_n = \frac{\Delta^n y}{n!(\Delta x)^n}\]
	\section{Characteristics of Polynomial Functions}
		\subsection{End Behaviour}
			For $x \to \pm\infty$ the graph of the polynomial function resembles the graph of the leading term $y=a_nx^n$.
			\vspace{\baselineskip}

			\begin{minipage}{0.5\textwidth}
				\centering
				For $a_n > 0$, $n$ is even
				\vspace{\baselineskip}

				\begin{tikzpicture}
					\draw[->] (-2,0) -- (2,0);
					\draw[->] (0,-2) -- (0,2);
					\draw[->] (-0.75,0.75) -- ++(-0.75,0.75);
					\draw[->] (0.75,0.75) -- ++(0.75,0.75);
				\end{tikzpicture}
			\end{minipage}
			\begin{minipage}{0.5\textwidth}
				\centering
				For $a_n > 0$, $n$ is odd
				\vspace{\baselineskip}

				\begin{tikzpicture}
					\draw[->] (-2,0) -- (2,0);
					\draw[->] (0,-2) -- (0,2);
					\draw[->] (-0.75,-0.75) -- ++(-0.75,-0.75);
					\draw[->] (0.75,0.75) -- ++(0.75,0.75);
				\end{tikzpicture}
			\end{minipage}
			\vspace{\baselineskip}

			\begin{minipage}{0.5\textwidth}
				\centering
				For $a_n < 0$, $n$ is even
				\vspace{\baselineskip}

				\begin{tikzpicture}
					\draw[->] (-2,0) -- (2,0);
					\draw[->] (0,-2) -- (0,2);
					\draw[->] (-0.75,-0.75) -- ++(-0.75,-0.75);
					\draw[->] (0.75,-0.75) -- ++(0.75,-0.75);
				\end{tikzpicture}
			\end{minipage}
			\begin{minipage}{0.5\textwidth}
				\centering
				For $a_n < 0$, $n$ is odd
				\vspace{\baselineskip}

				\begin{tikzpicture}
					\draw[->] (-2,0) -- (2,0);
					\draw[->] (0,-2) -- (0,2);
					\draw[->] (-0.75,0.75) -- ++(-0.75,0.75);
					\draw[->] (0.75,-0.75) -- ++(0.75,-0.75);
				\end{tikzpicture}
			\end{minipage}
		\subsection{Symmetry}
			A polynomial function is even ($f(-x)=f(x)$) if all the powers of $x$ are even.

			A polynomial function is odd ($f(-x)=-f(x)$) if all the powers of $x$ are odd.
		\subsection{Zeros and the Fundamental Theorem of Algebra}
			\begin{itemize}
				\item A polynomial function $P(x)$ of degree $n$ has $n$ zeros (real or complex).
				\item If the coefficients of the polynomial function are real numbers, the complex zeros come in conjugate pairs.
				\item The number of complex zeros must 0 or a multiple of 2.
				\item The number of real zeros is at most $n$. These zeros may be distinct or coincident.
				\item A polynomial function of even degree may have no real zero (all may be complex).
				\item A polynomial function of odd degree must have at least one real zero.
			\end{itemize}
		\subsection{Turning Points}
			A turning point is where a function changes from increasing to decreasing or decreasing to increasing.
			
			A polynomial function of degree $n$ has at most $n-1$ turning points.
		\subsection{Extrema Points}
			\begin{itemize}
				\item Extremum is either a minimum or a maximum.
				\item An extremum may be local or global.
				\item Each turning point is a local extremum.
				\item A polynomial function of even degree has either a global minimum or a global maximum point.
				\item A polynomial function of odd degree has neither a global minimum nor a global maximum point.
			\end{itemize}
	\section{Polynomial Functions in Factored Form}
		\subsection{Simple Zeroes}
			Some polynomial functions can be factored in the form:
			\[f(x) = a_n(x-x_1)(x-x_2)\dots(x-x_{n-1})(x-x_n)\]
			$x_1$, $x_2$, \dots , $x_{n-1}$, and $x_n$ are $n$ distinct real numbers and the zeros of the polynomial function.

			\begin{itemize}
				\item The function changes sign at each x-intercept.
				\item The tangent line at each x-intercept is not horizontal.
			\end{itemize}
		\subsection{Repeated Zeros}
			Some polynomial functions can be factored in the form:
			\[f(x)=a_n(x-x_1)^{m_1}(x-x_2)^{m_2}\dots(x-x_k)^{m_k}\]

			$x_1$ is a zero of multiplicity $m_1$, $x_2$ is a zero of multiplicity $m_2$, and so on.

			The polynomial function has
			\[\sum_{x=1}^{k}(m_x) = n\]
			real zeros.

			\begin{itemize}
				\item If $m_1$ is odd, the function changes sign at $x=x_1$ and the graph crosses the x-axis.
				\item If $m_1$ is even, the function does not change the sign at $x=x_1$ and the graph touches the x-axis.
				\item If the multiplicity $m_1 > 1$ then the tangent line at $x=x_1$ is horizontal. 
			\end{itemize}
		\subsection{Non-real Zeros}
			A polynomial function with non-real zeros can be factored as:
			\[f(x)=(a_1x^2+b_1x+c)^{m_1}\times\dots\]

			Each trinomial $a_1x^2+b_1x+c$ has the same sign for all real numbers $x$.
	\section{Transformations of Power Functions (Cubic, Quartic, and Other)}
		\subsection{Cubic Function}
			The cubic function has the parent $f(x)=x^3$ and after transformations may be written as:
			\[f(x)=a[b(x-c)^3]+d\]
			\vspace{\baselineskip}
			\begin{center}
				\begin{tikzpicture}
					\begin{axis}[
						width=\textwidth,
						xmin=-6,
						xmax=6,
						xtick style={draw=none},
						ytick style={draw=none},
						xticklabels={},
						yticklabels={},
					]
					\addplot[mystyle][
						domain=-5:5,
						samples=50,
					]
					{x^3};
					\end{axis}
				\end{tikzpicture}
			\end{center}
		\subsection{Quartic Function}
			The quartic function has the parent $f(x)=x^4$ and after transformations may be written as:
			\[f(x)=a[b(x-c)^3]+d\]
			\vspace{\baselineskip}
			\begin{center}
				\begin{tikzpicture}
					\begin{axis}[
						width=0.75\textwidth,
						xmin=-6,
						xmax=6,
						xtick style={draw=none},
						ytick style={draw=none},
						xticklabels={},
						yticklabels={},
					]
					\addplot[mystyle][
						domain=-5:5,
						samples=50,
					]
					{x^4};
					\end{axis}
				\end{tikzpicture}
			\end{center}
		\subsection{Power Function (Real Exponent)}
			The power function with a real exponent is defined by:
			\[f(x)=x^b \quad | \quad b \in \mathbb{R}\]
		\subsection{Power Function (Rational Exponent)}
			The power function with a rational exponent is defined by:
			\[f(x)=x^{\frac{m}{n}} \quad | \quad m \in \mathbb{Z} \wedge n \in \mathbb{Z}\]
	\section{Dividing Polynomials}
		\subsection{Division of Natural Numbers}
			If $D$ and $d \neq 0$ are two natural numbers, then there are unique numbers $q$ and $r$ such that the following is true:
			\[\frac{D}{d}=q+\frac{r}{d} \qquad\qquad\qquad \text{or} \qquad\qquad\qquad D=dq+r\] 
			\[0 \leq r < d\]
			where:
			\begin{itemize}
				\item $D$ is the dividend
				\item $d$ is the divisor
				\item $q$ is the quotient
				\item $r$ is the remainder
			\end{itemize}

			If $r=0$ then $D$ is divisible by $d$.
		\subsection{Division of Polynomials}
			If $D(x)$ and $d(x) \neq 0$ are two polynomial functions, then there are two unique polynomials $q(x)$ and $r(x)$ such that the following relation (called division statement) is true:
			\[\frac{D(x)}{d(x)}=q(x)+\frac{r(x)}{d(x)} \qquad\qquad\qquad \text{or} \qquad\qquad\qquad D(x)=d(x)q(x)+r(x)\] 
			\[0 \leq \text{degree}(r) < \text{degree}(d)\]
			Same rules as above.
		\newpage
		\subsection{Synthetic Division Algorithm}
			Synthetic division is a shorthand for dividing a polynomial $P(x)$ by a linear divisor $x-b$.
			\subsubsection{Example}
				\[\frac{-2x^3+3x^2-4x+5}{x-2}\]
				\vspace{\baselineskip}
				\begin{center}
					\polyhornerscheme[x=2]{-2x^3+3x^2-4x+5}
				\end{center}
				Here, the remainder is $-7$.
	\section{Factoring Polynomials}
		\subsection{The Remainder Theorem}
			If a polynomial $P(x)$ is divided by $x-b$ then the remainder is $r=P(b)$.
			\begin{proof}
				\begin{align*}
					P(x) &= q(x)(x-b)+r(x)\\
					P(b) &= q(b)(b-b)+r(b)\\
					P(b) &= r
				\end{align*}
			\end{proof}
		\subsection{The Remainder Theorem (II)}
			If a polynomial $P(x)$ is divided by $ax-b$ then the remainder is $r=P\left(\frac{b}{a}\right)$.
			\begin{proof}
				\begin{align*}
					P(x) &= q(x)(ax-b)+r(x)\\
					P\left(\frac{b}{a}\right) &= q\left(\frac{b}{a}\right)\left(\frac{ab}{a}-b\right)+r\left(\frac{b}{a}\right)\\
					P(b) &= r
				\end{align*}
			\end{proof}
		\subsection{The Factor Theorem}
			A polynomial $P(x)$ has $x-b$ as a factor if and only if $P(b)=0$.

			In this case $b$ is a zero of the polynomial function $P(x)$.
		\subsection{Integral Zero Theorem}
			If $x=b$ is an integral zero of the polynomial $P(x)$ with integral coefficients, then $b$ is a factor (divisor) of the constant term $a_0$ of the polynomial.
	\section{Factoring a Sum or Difference of Cubes (Powers)}
		\subsection{Difference of Cubes} 
			\[a^3-b^3=(a-b)(a^2+ab+b^2)\]
		\subsection{Sum of Cubes}
			\[a^3+b^3=(a+b)(a^2-ab+b^2)\]
		\subsection{Difference of Two Powers}
			For any natural number $n$, the following identity is true:
			\[a^n-b^n=(a-b)(a^{n-1}+a^{n-2}b+a^{n-3}b^2+\dots+a^2b^{n-3}+ab^{n-2}+b^{n-1})\]
		\subsection{Sum of Two Powers}
			If $n$ is an odd natural number, the following identity is true:
			\[a^n+b^n=(a+b)(a^{n-1}-a^{n-2}b+a^{n-3}b^2-\dots\pm a^2b^{n-3} \mp ab^{n-2} \pm b^{n-1}\]
