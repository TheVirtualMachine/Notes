\chapter{Rational Functions}
	\section{Graphs of Reciprocal Functions}
		\subsection{General Rules}
			Consider a function $y=f(x)$ and its reciprocal $g(x) = \frac{1}{f(x)}$. Here are some general rules:
			\begin{itemize}
				\item If $f(x)$ has a zero at $x=a$, then $g(x)$ has a vertical asymptote $x=a$
				\item If $f(x)$ has a vertical asymptote $x=a$ ($y \to \pm\infty \text{ as } x \to a$), then $g(x)$ has a zero at $x=a$
				\item If $f(x)$ is unbounded as $x$ becomes unbounded ($y \to \pm\infty \text{ as } x \to \pm\infty$) then $g(x)$ has the horizontal asymptote $y=0$
				\item If $f(x)$ has a horizontal asymptote $y=a$ ($y \to a \text{ as } x \to \pm\infty$), then $g(x)$ has a horizontal asymptote $y=\frac{1}{a}$
				\item If $f(x)$ is increasing/decreasing over an interval, then $g(x)$ is decreasing/increasing over the same interval
				\item If $f(x)$ has a local minimum/maximum at $(a, f(a))$, then $g(x)$ has a local maximum/minimum at $(a, g(a))$
				\item If $f(x)$ is even/odd/neither, then so is $g(x)$
			\end{itemize}
	\section{Exploring Quotients of Polynomial Functions (Rational Functions)}
		\subsection{Rational Functions}
			A rational function is a function of the form:
			\[f(x)=\frac{P(x)}{Q(x)}\]
			where $P(x)$ and $Q(x)$ are polynomial functions.
		\subsection{Domain}
			The domain of a rational function is determined by the restriction $Q(x) \neq 0$.
		\subsection{y-intercept}
			The y-intercept for a rational function $f(x)$ is the point $(0, f(0))$ if 0 is in the domain of $f$.
		\subsection{Holes}
			The $x$ values where $Q(x) = 0$ do not belong to the domain of the function and must not appear on the graph of a rational function. These values might create holes in the graph of the rational function.
		\newpage
		\subsection{Vertical Asymptotes}
			If $x=a$ is a vertical asymptote, then the value of the function becomes unbounded as $x \to a$ from the left or right.
			\\
			In short, $x=a$ is a vertical asymptote if:
			\[\lim_{x \to a-} f(x) = f(a - h) = \pm\infty\]
			or
			\[\lim_{x \to a+} f(x) = f(a + h) = \pm\infty\]
			where $h > 0 \wedge h \text{ is very small}$

			If $x=a$ is a vertical asymptote for a rational function $f(x) = \frac{P(x)}{Q(x)}$, then $a$ is a zero of $Q(x)$ and not a zero of $P(x)$.
		\subsection{Behaviour Near the Vertical Asymptote}
			The value of the rational function is unbounded as $x \to a$ from the left and the right. To find the unbounded value of the rational function, use substitution, $x = a \pm h$ and the following types of limits:
			\[\frac{1}{h}=\infty \qquad \frac{1}{-h}=-\infty \qquad \frac{1}{h^2}=\infty \qquad \frac{1}{(-h)^2}=\infty \qquad \frac{1}{(-h)^3}=-\infty\]
			and so on.
		\subsection{Horizontal Asymptotes}
			If $y=c$ is a horizontal asymptote, then $f(x) \to c \text{ as } x \to \pm\infty$.
			\\
			In short, $y=c$ is a horizontal asymptote if:
			\[\lim_{x \to \infty} f(x) = c\]
			or
			\[\lim_{x \to -\infty} f(x) = c\]
			where $c$ is a finite number.

			Some functions may have two different horizontal asymptotes (one as $x \to \infty$ and one as $x \to -\infty$). Rational functions have at most one horizontal asymptote.

			Consider the case of the rational function:
			\[f(x)=\frac{P(x)}{Q(x)}=\frac{a_nx^n + \dots + a_1x + a_0}{b_mx^m + \dots + b_1x + a_0}\]
			\begin{itemize}
				\item If $P(x)$ and $Q(x)$ have the same degree ($n=m$), then the equation of the horizontal asymptote is $y=\frac{a_n}{b_m}$.
				\item If the degree of $P(x)$ is less than the degree of $Q(x)$, then the equation of the horizontal asymptote is $y=0$.
				\item If the degree of $P(x)$ is greater than the degree of $Q(x)$ then the rational function does not have a horizontal asymptote.
			\end{itemize}
		\subsection{Oblique Asymptotes}
			A rational function has an oblique asymptote if the degree of $P(x)$ is one unit greater than the degree of $Q(x)$. To find the equation of the oblique asymptote use long division and express the rational function in the form:
			\[f(x)=mx+b+\frac{c}{R(x)}\]
			As $|x| \to \pm\infty$, the term $\frac{c}{R(x)}$ approaches 0 and $f(x)$ approaches $y=mx+b$.
		\subsection{Graph Sketching}
			We have these tools to help us sketch the graphs of functions:
			\begin{tasks}[style=itemize](2)
				\task x-intercepts and y-intercepts
				\task symmetry
				\task vertical asymptotes
				\task horizontal asymptotes
				\task oblique asymptotes
				\task sign charts
			\end{tasks}
	\section{Graphs of Rational Functions}
		\subsection{Characteristics of the Rational Function}
			\[f(x)=\frac{ax+b}{cx+d} \quad | \quad a, c \neq 0\]
			\subsubsection{Case 1: $cx+d$ is not a factor of $ax+b$}
				\begin{description}
					\item[Domain] $\mathbb{R} \setminus \{-\frac{d}{c}\}$
					\item[Range] $\mathbb{R} \setminus \{\frac{a}{c}\}$
					\item[x-intercept] $-\frac{b}{a}$
					\item[y-intercept] $\frac{b}{d}$ if $d \neq 0$
					\item[Symmetry] neither even nor odd
					\item[Vertical asymptote] $x=-\frac{d}{c}$
					\item[Horizontal asymptote] $y={a}{c}$
					\item[Continuity] There exists an infinite break at $x=-\frac{d}{c}$
				\end{description}
			\subsubsection{Case 2: $cx+d$ is a factor of $ax+b$}
				\begin{description}
					\item[Domain] $\mathbb{R} \setminus \{-\frac{d}{c}\}$
					\item[Range] $\{\frac{a}{c}\}$
					\item[x-intercept] none
					\item[y-intercept] $\frac{b}{d}$ if $d \neq 0$
					\item[Symmetry] neither even nor odd
					\item[Vertical asymptote] none
					\item[Horizontal asymptote] $y=\frac{a}{c}$
					\item[Continuity] There exists a hole at $x=-\frac{d}{c}$
				\end{description}
	\section{Solving Rational Equations}
		\subsection{Rational Equations}
			To solve a rational equation:
			\begin{enumerate}
				\item State restrictions
				\item Multiply by the least common denominator
				\item Solve the polynomial equation
				\item Verify restrictions
				\item Verify solution by using substitution
			\end{enumerate}
		\subsection{Cross Multiplication}
			A rational equation of the form: $\frac{P(x)}{Q(x)} = \frac{R(x)}{S(x)}$, where $P(x)$, $Q(x)$, $R(x)$, and $S(x)$ are polynomial functions may be solved with cross-multiplication.
			\[\frac{P(x)}{Q(x)} = \frac{R(x)}{Q(x)} \qquad \equiv \qquad P(x)S(s) = Q(x)R(x)\]
		\subsection{Shortcut}
			For a rational equation of the form $\frac{P(x)}{Q(x)} = 0$:
			\[\frac{P(x)}{Q(x)} = 0 \quad \equiv \quad P(x) = 0 \qquad \text{if restrictions are satisfied}\]
		\subsection{No Solution}
			When $c$ is a constant:
			\[\frac{\text{c}}{P(x)} = 0 \quad \text{has no solution} \quad \iff \quad c \neq 0\]
	\section{Solving Rational Inequalities}
		\subsection{Rational Inequalities}
			In order to solve a nonlinear rational inequality:
			\begin{enumerate}
				\item State restrictions
				\item Move all the terms to one side
				\item Multiply the expression by the least common denominator
				\item Simplify and factor both numerator and denominator
				\item Create a sign chart or graph to find the solution set
				\item Verify if restrictions are satisfied
			\end{enumerate}
	\section{Component Partial Fraction Decomposition}
		\subsection{Partial Fractions}
			The process of breaking down a rational expression such as $\frac{3x+7}{(x+5)(x+1)}$ into partial fractions such as $\frac{2}{x+5} + \frac{1}{x+1}$ is called partial fraction decomposition.

			The rational function $R(x) = \frac{P(x)}{Q(x)}\ |\ Q(x) \neq 0$ is called a proper fraction when the degree of $P(x)$ is equal to the degree of $Q(x)$. We assume that $P(x)$ and $Q(x)$ have no common zeros.

			Partial fraction decomposition depends on the factors in the denominator $Q(x)$. 
			
			When $Q(x)$ is factored as a product of $(ax+b)^n$ and $(ax^2+bx+c)^m$:
			\[n \in \mathbb{N^*} \qquad m \in \mathbb{N^*} \qquad a \in \mathbb{R} \qquad b \in \mathbb{R} \qquad c \in \mathbb{R}\]
			and $(ax^2+bx+c)$ is irreducible over the real numbers.
			\newpage
			\subsubsection{Case 1: $Q(x)$ contains only distinct linear factors}
				\[Q(x) = (a_1x+b_1)(a_2x+b_2)\dots(a_nx+b_n)\]
				In this case, unique real constants $C_1$, $C_2$, \dots , $C_n$ could be found such that:
				\[\frac{P(x)}{Q(x)} = \frac{C_1}{a_1x+b_1} + \frac{C_2}{a_2x+b_2} + \dots + \frac{C_n}{a_nx+b_n}\]
				\paragraph{Example}
					Decompose $\frac{2x+1}{(x-1)(x+3)}$ into partial fractions.
					\subparagraph{Step 1}
						\[\frac{2x+1}{(x-1)(x+3)} = \frac{A}{x-1} + \frac{B}{x+3}\]
					\subparagraph{Step 2}
						Find the LCD of the right side and equate the numerators:
						\[2x+1 = A(x+3) + B(x-1)\]
					\subparagraph{Step 3}
						Find A and B by equating all coefficients of all powers of $x$:

						For the coefficient of $x^0$:
						\[1 = 3A - B\]
						For the coefficient of $x^1$:
						\[2 = A + B\]

						Solve the system to find that:
						\[A = \frac{3}{4} \qquad B = \frac{5}{4}\]
					\subparagraph{Rewrite the rational expression as a sum of two partial fractions}
						\[\frac{2x+1}{(x-1)(x+3)} = \frac{3}{4(x-1)} + \frac{5}{4(x+1)}\]
			\newpage
			\subsubsection{Case 2: $Q(x)$ contains only one repeated linear factor}
				$Q(x)$ contains a repeated factor of $(ax+b)$ such that $Q(x) = (ax+b)^n$.

				In this case, the partial fraction decomposition of $R(x)$ could be written as:
				\[R(x) = \frac{C_1}{(ax+b)} + \frac{C_2}{(ax+b)^2)} + \dots + \frac{C_n}{(ax+b)^n}\]
				\paragraph{Example}
					Decompose $\frac{6x-1}{x^3(2x-1)}$ into partial fractions.
					\subparagraph{Step 1}
						Write $R(x)$ according to case 1 and case 2 as:
						\[\frac{6x-1}{x^3(2x-1)} = \frac{A}{x} + \frac{B}{x^2} + \frac{C}{x^3} + \frac{D}{2x-1}\]
					\subparagraph{Step 2}
						Multiply the expression by $x^3(2x-1)$ and equate the numerators from both sides:
						\[6x-1 = Ax^2(2x-1) + Bx(2x-1) + C(2x-1) + Dx^3\]
					\subparagraph{Step 3}
						Solve for $A$, $B$, $C$, and $D$ by equating all coefficients of all powers of $x$:
						
						\begin{minipage}[t]{0.5\textwidth}
							\centering
							For the coefficient of $x^0$:
							\[-1 = -C\]
							For the coefficient of $x^1$:
							\[6 = -B + 2C\]
						\end{minipage}
						\begin{minipage}[t]{0.5\textwidth}
							\centering
							For the coefficient of $x^2$:
							\[0 = -A + 2B\]
							For the coefficient of $x^3$:
							\[0 = 2A + D\]
						\end{minipage}

						Now solve the system:

						\begin{minipage}[t]{0.5\textwidth}
							\begin{align*}
								-1 &= C \\
								C &= 1
							\end{align*}
							\begin{align*}
								6 &= -B + 2 \tag{Substitute in $C=1$}\\
								B &= -4
							\end{align*}
						\end{minipage}
						\begin{minipage}[t]{0.5\textwidth}
							\begin{align*}
								0 &= -A - 8 \tag{Substitute in $B=-4$}\\
								A &= -8
							\end{align*}
							\begin{align*}
								0 &= -16 + D \tag{Substitute in $A=-8$}\\
								D &= 16
							\end{align*}
						\end{minipage}
					\subparagraph{Step 4}
						Rewrite the rational expression as a sum of the partial fractions:
						\[\frac{6x-1}{x^3(2x-1)} = -\frac{8}{x} - \frac{4}{x^2} + \frac{1}{x^3} + \frac{16}{2x-1}\]
			\newpage
			\subsubsection{Case 3: $Q(x)$ contains only nonrepeated irreducible quadratic factors}
				$(a_ix+b_ix+c)$ are among the factors of $Q(x)$.

				In this case, the partial fraction decomposition of $R(x)$ could be written as:
				\[R(x) = \frac{A_1x+B_1}{(a_1x^2+b_1x+c_1)} + \frac{A_2x+B_2}{(a_2x^2+b_2x+c_2)} + \dots + \frac{A_nx+B_n}{a_nx^2+b_nx+c_n}\]
				\paragraph{Example}
					Decompose $\frac{4x}{(x^2+1)(x^2+2x+3)}$ into partial fractions.
					\subparagraph{Step 1}
						Write $R(x)$ according to case 3 as:
						\[\frac{4x}{(x^2+1)(x^2+2x+3)} = \frac{Ax+B}{x^2+1} + \frac{Cx+D}{x^2+2x+3}\]
					\subparagraph{Step 2}
						Multiply the expression by $(x^2+1)(x^2+2x+3)$ and equate the numerators from both sides:
						\[4x = (Ax+B)(x^2+2x+3) + (Cx+D)(x^2+1)\]
					\subparagraph{Step 3}
						Solve for $A$, $B$, $C$, and $D$ by equating all coefficients of all powers of $x$:

						\begin{minipage}[t]{0.5\textwidth}
							\centering
							For the coefficient of $x^0$:
							\[0 = 3B + D\]
							For the coefficient of $x^1$:
							\[4 = 3A + 2B + C\]
						\end{minipage}
						\begin{minipage}[t]{0.5\textwidth}
							\centering
							For the coefficient of $x^2$:
							\[0 = 2A + B + D\]
							For the coefficient of $x^3$:
							\[0 = A + C\]
						\end{minipage}

						Solve the system:
						\\
						\begin{minipage}[t]{0.2\textwidth}
							\begin{align*}
								3B + D &= 2A + B + D \\
								B &= A
							\end{align*}
							\begin{equation*}
								\begin{array}{lll}
									  &4 &= 5A + C \\
									- &0 &=  A + C \\\hline
									  &4 &= 4A \\
								\end{array}
							\end{equation*}
							\[A=1 \quad B=1\]
						\end{minipage}
						\begin{minipage}[t]{0.8\textwidth}
							\begin{align*}
								0 &= A+C \\
								C &= -1 \tag{Substitute in $A=1$}
							\end{align*}
							\begin{align*}
								0 &= 2A + B + D \\
								D &= -3 \tag{Substitute in $A=1$ and $B=1$}
							\end{align*}
						\end{minipage}
					\subparagraph{Step 4}
						Rewrite the rational expression as a sum of the partial fractions:
						\[\frac{4x}{(x^2+1)(x^2+2x+3)} = \frac{x+1}{x^2+1} - \frac{x+3}{x^2+2x+3}\]
