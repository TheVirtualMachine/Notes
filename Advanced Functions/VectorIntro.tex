\chapter{Introduction to Vectors}
	\section{An Introduction to Vectors}
		\subsection{Scalars and Vectors}
			Scalars are described with a number.

			Vectors are described with magnitude and direction.
		\subsection{Geometric Vectors}
			Geometric vectors are vectors not related to any coordinate system.

			For example: $\vv{AB}$
		\subsection{Algebraic Vectors}
			Algebraic vectors are vectors related to a coordinate system.

			For example: $\vv{v} = (2, 3, -1)$
		\subsection{Position Vector}
			The position vector is the directed line segment $\vv{OP}$ from the origin of the coordinate system $O$ to a generic point $P$.
		\subsection{Displacement Vector}
			The displacement vector $\vv{AB}$ is the directed line segment from the point $A$ to the point $B$.
		\subsection{Pythagorean Theorem}
			In a right triangle $ABC$ with $\angle C = \ang{90}$, the following relation is true:
			\[c^2 = a^2 + b^2\]
		\subsection{Magnitude}
			The magnitude of $\vv{v}$ is denoted by $|\vv{v}|$, $\|\vv{v}\|$, or $v$.
		\subsection{3D Pythagorean Theorem}
			\[d^2 = a^2 + b^2 + c^2\]
		\subsection{Equivalent or Equal Vectors}
			Two vectors are equivalent or equal if they have the same magnitude and direction.
		\subsection{Opposite Vectors}
			Two vectors are called opposite if they have the same magnitude and opposite direction.

			The vector opposite $\vv{v}$ is denoted by $-\vv{v}$. The vector opposite $\vv{AB}$ is $\vv{BA}$.

			Also, $\vv{AB} = -\vv{BA}$.
		\subsection{Parallel Vectors}
			Two vectors are parallel if their directions are either the same or opposite.

			If $\vv{v_1}$ and $\vv{v_2}$ are parallel, then we write $\vv{v_1} \parallel \vv{v_2}$.
		\subsection{Direction}
			True bearing is measured clockwise from North.

			Quadrant bearing is given by the angle between the North-South line and the vector.
	\section{Addition and Subtraction of Geometric Vectors}
		\subsection{Addition of Two Vectors}
			The vector addition $\vv{s}$ of two vectors $\vv{a}$ and $\vv{b}$ is denoted by $\vv{a}+\vv{b}$ and is called the sum or resultant of the two vectors. So:
			\[\vv{s} = \vv{a} + \vv{b}\]
		\subsection{Triangle Rule (Tail to Tip Rule)}
			In order to find the resultant of two geometric vectors:
			\begin{enumerate}
				\item Place the second vector with its tail on the tip of the first vector.
				\item The resultant is a vector with the tail at the tail of the first vector and the head at the head of the second vector.
			\end{enumerate}
		\subsection{Polygon Rule}
			In order to find the resultant of $n$ geometric vectors:
			\begin{enumerate}
				\item Place the next vector with its tail on the tip of the previous vector.
				\item The resultant is a vector with the tail at the tail of the fist vector and the head at the head of the last vector.
			\end{enumerate}
		\subsection{Parallelogram Rule (Tail to Tail Rule)}
			To add two geometric vectors, the following rule can also be used:
			\begin{enumerate}
				\item Position both vectors with their tails at the same point.
				\item Build a parallelogram using the vectors as two sides.
				\item The resultant is given by the diagonal of the parallelogram starting from the common tail point.
			\end{enumerate}
		\subsection{Sine Law}
			For any triangle $\triangle ABC$, the following relation is true:
			\[\frac{\sin A}{a} = \frac{\sin B}{b} = \frac{\sin C}{c} \qquad \equiv \qquad \frac{a}{\sin A} = \frac{b}{\sin B} = \frac{c}{\sin C}\]
		\subsection{Cosine Law}
			For any triangle $\triangle ABC$, the following relations are true:
			\[a^2 = b^2 + c^2 - 2bc\cos\angle A\]
			\[b^2 = a^2 + c^2 - 2ac\cos\angle B\]
			\[c^2 = a^2 + b^2 - 2ab\cos\angle C\]
		\subsection{Magnitude and Direction for Vector Sum}
			Let $\theta = \angle(\vv{a}, \vv{b})$ be the angle between the vectors $\vv{a}$ and $\vv{b}$ when they are placed tail to tail.
		\subsection{Vector Subtraction}
			The subtraction operation between two vectors $\vv{a} - \vv{b}$ can be understood as a vector addition between the first vector and the opposite of the second vector:
			\[\vv{d} = \vv{a} - \vv{b} = \vv{a} + (-\vv{b})\]
		\subsection{Inverse Operation}
			The vector subtraction operation is the inverse operation of the vector addition:
			\[\vv{d} = \vv{a} - \vv{b} \quad \equiv \quad \vv{a} = \vv{b} + \vv{d}\]
		\subsection{Magnitude and Direction for Vector Difference}
			Let $\vv{a}$ and $\vv{b}$ be two vectors and $\vv{d} = \vv{a} - \vv{b}$ be the vector difference. Let $\theta$ be the angle between the vectors $\vv{a}$ and $\vv{b}$ when they are placed tail to tail:
			\[\vv{d} = \vv{a} - \vv{b}\]
			The magnitude of the vector difference is given by:
			\[\|\vv{a} - \vv{b}\|^2 = \|\vv{a}\|^2 + \|\vv{b}\|^2 - 2\|\vv{a}\|\,\|\vv{b}\|\cos \theta\]
			The direction of $\vv{d}$ is given by the angles $\alpha$ and $\beta$ formed by the vector sum and the vectors $\vv{b}$ and $\vv{a}$ respectively:
			\[\frac{\|\vv{a}\|}{\sin\alpha} = \frac{\|\vv{b}\|}{\sin\beta} = \frac{\|\vv{a}-\vv{b}\|}{\sin\theta}\]
	\section{Multiplication of a Vector by a Scalar}
		\subsection{Multiplication of a Vector by a Scalar}
			By multiplying a vector $\vv{v}$ by a scalar $k$, we obtain a new vector noted $k\vv{v}$ with the following properties:
			\begin{itemize}
				\item $k\vv{v}$ has the same direction as $\vv{v}$ if $k > 0$ and the opposite direction if $k < 0$
				\item $\|k\vv{v}\| = |k| \times \|\vv{v}\|$
			\end{itemize}
		\subsection{Properties}
			The following properties apply for multiplication of a vector by a scalar:
			\[k(\vv{a} + \vv{b}) = k\vv{a} + k\vv{b}\]
			\[k(m\vv{a}) = (km)\vv{a} = km\vv{a}\]
			\[(k + m)\vv{a} = k\vv{a} + m\vv{a}\]
		\subsection{Vector Unit}
			A unit vector is a vector having a magnitude of 1. For any vector $\vv{v}$, a unit vector parallel to $\vv{v}$ is given by:
			\[\hat{v} = \frac{\vv{v}}{\|\vv{v}\|}\]
	\section{Properties of Vectors}
		\subsection{Properties of Vectors}
			\[\vv{a} + \vv{b} = \vv{b} + \vv{a}\]
			\[\vv{a} + \vv{0} = \vv{0} + \vv{a} + \vv{a}\]
			\[\vv{a} + (-\vv{a}) = (-\vv{a}) + \vv{a} = \vv{0}\]
			\[(\vv{a} + \vv{b}) + \vv{c} = \vv{a} + (\vv{b} + \vv{c})\]
			\[\|k\vv{a}\| = |k|\,\|\vec{a}\|\]
			\[k(\vv{a}+\vv{b}) = k\vv{a} + k\vv{b}\]
			\[(kl)\vv{a} = k(l\vv{a}) = l(k\vv{a})\]
			\[(k + l)\vv{a} = k\vv{a} + l\vv{a}\]
			\[1\vv{a} = \vv{a}\]
			\[(-1)\vv{a} = -\vv{a}\]
			\[0\vv{a} = \vv{0}\]
			\[\|\vv{0}\| = 0\]
	\section{Vectors in $R^2$ and $R^3$}
		\subsection{Polar Coordinates}
			Given a Cartesian system of coordinates, 2D vector $\vv{v}$ may be defined by its magnitude $\|\vv{v}\|$ and the counter-clockwise angle $\theta$ between the positive direction of the x-axis and the vector.

			The pair $(\|\vv{v}\|, \theta)$ determines the polar coordinates of the 2D vector and $\vv{v} = (\|\vv{v}\|, \theta)$.
		\subsection{Scalar Components for a 2D Vector}
			Consider a 2D vector with the tail in the origin of the Cartesian system. Parallels through its tip to the coordinate axes intersect the x-axis at $v_x$ and the y-axis at $v_y$.

			The pair $(v_x, v_y)$ determines the scalar coordinates of the 2D vector and $\vv{v} = (v_x, v_y)$.
		\subsection{Link Between the Polar Coordinates and Scalar Components}
			To convert a vector from the polar coordinates $\vv{v} = (\|\vv{v}\|, \theta)$ to the scalar components $\vv{v}=(v_x, v_y)$, use the formulas:
			\[v_x = \|\vv{v}\|\cos\theta\]
			\[v_y = \|\vv{v}\|\sin\theta\]
			To convert a vector from the scalar components $\vv{v}=(v_x, v_y)$ to the polar coordinates $\vv{v}=(\|\vv{v}\|, \theta)$, use the formulas:
			\[\|\vv{v}\| = \sqrt{{v_x}^2 + {v_y}^2}\]
			\[\tan\theta = \frac{v_y}{v_x} \qquad \equiv \qquad \theta = \tan^{-1} \left(\frac{v_y}{v_x}\right)\]
		\subsection{Magnitude of a 2D Algebraic Vector}
			The magnitude of a 2D algebraic vector $\vv{v}=(v_x, v_y)$ is given by:
			\[\|\vv{v}\| = \sqrt{{v_x}^2 + {v_y}^2}\]
		\subsection{Standard Unit Vectors}
			\[\vv{i} = (1, 0)\]
			\[\vv{j} = (0, 1)\]
		\subsection{Vector Components for a 2D Vector}
			Any vector $\vv{v}$ may be decomposed into two $\perp$ vector components $\vv*{v}{x}$ and $\vv*{v}{y}$, where:
			\[\vv*{v}{x} \parallel \vv{i} \qquad \vv*{v}{y} \parallel \vv{j}\]
			\[\vv{v} = \vv*{v}{x} + \vv*{v}{y}\]
			The link between the scalar components and the vector components is given by:
			\[\vv*{v}{x} = v_x\vv{i} \qquad \vv*{v}{y} = v_y\vv{j}\]
			A 2D vector may also be written in algebraic form as:
			\[\vv{v} = \vv*{v}{x} + \vv*{v}{y} = v_x\vv{i} + v_y\vv{j} = (v_x, v_y)\]
		\subsection{Position 2D Vector}
			The directed line segment $\vv{OP}$, from the origin $O$ to a generic point $P(x, y)$ determines a vector called the position vector and:
			\[\vv{OP} = (x, y)  = x\vv{i} + y\vv{j}\]
		\subsection{Displacement 2D Vector}
			The directed line segment $\vv{AB}$ from the point $A(x_A, y_a)$ to the point $B(x_B, y_B)$ determines a vector called the displacement vector and:
			\[\vv{AB} = (x_B - x_A, y_B - y_A) = (x_B-x_A)\vv{i} + (y_B - y_A)\vv{j}\]
		\subsection{Direction Angles}
			Consider a 3D coordinate system and a 3D vector $\vv{v}$ with the tail in the origin $O$. Direction angles are the angles $\alpha$, $\beta$, and $\gamma$ between the vector and the positive directions of the coordinate axes.
			\begin{description}
				\item[$\alpha$:] the angle between the vector and the x-axis.
				\item[$\beta$:] the angle between the vector and the y-axis.
				\item[$\gamma$:] the angle between the vector and the z-axis.
			\end{description}
		\subsection{Scalar Components of a 3D Vector}
			Consider a 3D coordinate system and a 3D vector $\vv{v}$ with the tail in the origin $O$. Parallel planes through its tip to the coordinate planes intersect the x-axis at $v_x$, the y-axis at $v_y$, and the z-axis at $v_z$.

			The triple $(v_x, v_y, v_z)$ determines the scalar components of the 3D vector and $\vv{v} = (v_x, v_y, v_z)$.
		\subsection{Link Between the Direction Angles and the 3D Scalar Coordinates}
			The link between the direction angles ($\alpha$, $\beta$, and $\gamma$) and the scalar components of a vector ($v_x$, $v_y$, and $v_z$) is given by:
			\[v_x = \|\vv{v}\|\cos\alpha\]
			\[v_y = \|\vv{v}\|\cos\beta\]
			\[v_z = \|\vv{v}\|\cos\gamma\]
			and by:
			\[\|\vv{v}\| = \sqrt{{v_x}^2 + {v_y}^2 + {v_z}^2}\]
			\[\cos\alpha = \frac{v_x}{\|\vv{v}\|}\]
			\[\cos\beta = \frac{v_y}{\|\vv{v}\|}\]
			\[\cos\gamma = \frac{v_z}{\|\vv{v}\|}\]
			Note that:
			\[\cos^2\alpha + \cos^2\beta + \cos^2\gamma = 1\]
		\subsection{Magnitude of a 3D Algebraic Vector}
			The magnitude of a 3D algebraic vector $\vv{v} = (v_x, v_y, v_z)$ is given by:
			\[\|\vv{v}\| = \sqrt{{v_x}^2 + {v_y}^2 + {v_z}^2}\]
		\subsection{3D Standard Unit Vectors}
			\[\vv{i} = (1,0,0)\]
			\[\vv{j} = (0,1,0)\]
			\[\vv{k} = (0,0,1)\]
		\subsection{Vector Components for a 3D Vector}
			Any 3D vector $\vv{v}$ may be decomposed into three $\perp$ vector components $\vv*{v}{x}$, $\vv*{v}{y}$, and $\vv*{v}{z}$ where:
			\[\vv*{v}{x} \parallel \vv{i} \qquad \vv*{v}{y} \parallel \vv{j} \qquad \vv*{v}{z} \parallel \vv{k}\]
			\[\vv{v} = \vv*{v}{x} + \vv*{v}{y} + \vv*{v}{z}\]
			The link between the scalar components and the vector components is given by:
			\[\vv*{v}{x} = v_x\vv{i} \qquad \vv*{v}{y} = v_y\vv{j} \qquad \vv*{v}{z} = v_z\vv{k}\]
			A 3D vector may be written in algebraic form as:
			\[\vv{v} = \vv*{v}{x} + \vv*{v}{y} + \vv*{v}{z} = v_x\vv{i} + v_y\vv{j} + v_z\vv{k} = (v_x, v_y, v_z)\]
		\subsection{Position 3D Vector}
			The directed line segment $\vv{OP}$ from the origin $O$ to a generic point $P(x, y, z)$ determines a vector called the position vector and:
			\[\vv{OP} = (x, y, z) = x\vv{i} + y\vv{j} + z\vv{k}\]
		\subsection{Displacement 3D Vector}
			The directed line segment $\vv{AB}$ from the point $A(x_A, y_A, z_A)$ to the point $B(x_B, y_B, z_B)$ determines a vector called the displacement vector and:
			\[\vv{AB} = (x_B - x_A, y_B - y_A, z_B - z_A)\]
			\[\vv{AB} = (x_B - x_A)\vv{i} + (y_B - y_A)\vv{j} + (z_B - z_A)\vv{k}\]
	\section{Operations with Algebraic Vectors in $R^2$}
		\subsection{2D Algebraic Vectors}
			A 2D algebraic vector may be written in components form as:
			\[\vv{v} = (v_x, v_y)\]
			or in terms of unit vectors as:
			\[\vv{v} = v_x\vv{i} + v_y\vv{j}\]
			and has a magnitude given by:
			\[\|\vv{v}\| = \sqrt{{v_x}^2 + {v_y}^2}\]
		\subsection{Addition of 2D Algebraic Vectors}
			\[\vv{a} + \vv{b} = (a_x + b_x, a_y + b_y)\]
		\subsection{Substitute of 2D Algebraic Vectors}
			\[\vv{a} - \vv{b} = (a_x - b_x, a_y - b_y)\]
		\subsection{Multiplication of 2D Algebraic Vector by a Scalar}
			\[\lambda\vv{a} = (\lambda a_x, \lambda a_y)\]
		\subsection{Vector Equations}
			Use backward operations to solve equations using vectors:
			\[\vv{x} + \vv{a} = \vv{b} \implies \vv{x} = \vv{b} - \vv{a}\]
			\[\vv{a} - \vv{x} = \vv{b} \implies \vv{x} = \vv{a} - \vv{b}\]
			\[\lambda\vv{x} = \vv{a} \implies \vv{x} = \frac{\vv{a}}{\lambda}\]
	\section{Operations with Algebraic Vectors in $R^3$}
		\subsection{3D Algebraic Vectors}
			A 3D algebraic vector may be written as:
			\[\vv{v} = (v_x, v_y, v_z) \qquad \equiv \qquad \vv{v} = v_x\vv{i} + v_y\vv{j} + v_z\vv{k}\]
			and has a magnitude given by:
			\[\|\vv{v}\| = \sqrt{{v_x}^2 + {v_y}^2 + {v_z}^2}\]
		\subsection{Addition of 3D Algebraic Vectors}
			\[\vv{a} + \vv{b} = (a_x + b_x, a_y + b_y, a_z + b_z)\]
		\subsection{Subtraction of 3D Algebraic Vectors}
			\[\vv{a} - \vv{b} = (a_x - b_x, a_y - b_y, a_z - b_z)\]
		\subsection{Multiplication of 3D Algebraic Vector by a Scalar}
			\[\lambda\vv{a} = (\lambda a_x, \lambda a_y, \lambda a_z)\]
		\subsection{Midpoint}
			The midpoint of the segment line $\vv{AB}$ is the point $M$ such that $\vv{MA} + \vv{MB} = \vv{0}$
		\subsection{Centroid}
			The centroid of a system of points $P_1, P_2, \dots, P_n$ is the point $C$ defined by:
			\[\vv{OC} = \frac{\vv*{OP}{1} + \vv*{OP}{2} + \dots + \vv*{OP}{n}}{n}\]
		\subsection{Parallelism}
			$\vv{a} \parallel \vv{b}$ if there exists $\lambda$ such that $\vv{a} = \lambda\vv{b}$.

			$\vv{a}$ and $\vv{b}$ may have the same direction or the opposite direction.
		\subsection{Co-linearity}
			Three points $A$, $B$, and $C$ are collinear if $\vv{AB} \parallel \vv{BC}$.
		\subsection{Linear Dependency}
			$\vv{a}$, $\vv{b}$, and $\vv{c}$ are linear dependent if there exists $\lambda$ and $\mu$ such that $\vv{c} = \lambda\vv{a} + \mu\vv{b}$.

			The vectors must be coplanar for this to ever be true.
