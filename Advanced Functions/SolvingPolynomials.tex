\chapter{Solving Polynomial and Linear Equations and Inequalities}
	\section{Solving Polynomial Equations}
		\subsection{The Fundamental Theorem of Algebra}
			A polynomial function $P(x)$ of degree $n$ has $n$ zeros (real or complex).

			The complete factorization of the polynomial function $P(x)$ is:
			\[P(x)=a_n(x-x_1)(x-x_2)\dots(x-x_{n-1})(x-x_n)\]
			where $x_1$, $x_2$, \dots, $x_{n-1}$, and $x_n$ are the zeros (real or complex, distinct or coincident).

			The complex zeros come in conjugate pairs. So:
			\[n=r+2m \quad | \quad m \geq 0\]
			where $r$ is the number of real zeros and $2m$ is the number of complex zeros.
		\subsection{Integral Zero Theorem}
			If $x=b$ is an integral zero of the polynomial $P(x)$ with integral coefficients, then $b$ is a factor of the constant term $a_0$ of the polynomial.
		\subsection{Rational Zero Theorem}
			If $x=\frac{b}{a}$ is a rational zero of the polynomial $P(x)$ with rational coefficients then $b$ is a factor (divisor) of the constant term $a_0$ and $a$ is a factor of the leading coefficient $a_n$.
		\subsection{Real Zeros}
			If $x=a+b\sqrt{c}$ ($a$, $b$, and $c$ are rational numbers, $c>0$) is a zero of a polynomial with integral coefficients $P(x)$, then $x=a-b\sqrt{c}$ is also a zero of this polynomial.
		\subsection{Technology}
			Is some cases the real zeros may only be found by using technology.
	\section{Solving Linear Inequalities}
		\subsection{Inequalities}
			The inequality symbols are used to create inequalities. They are:
			\[< \qquad \leq \qquad \geq \qquad >\]
			
			The solution set is all numbers that make the inequality a true statement.
		\subsection{Inequality Properties}
			The inequality $a<b$ is equivalent to:
			\begin{align*}
				a + c &< b + c\\
				ac &< bc \qquad\qquad \text{for } c>0\\
				ac &> bc \qquad\qquad \text{for } c<0\\
			\end{align*}
		\subsection{Simultaneous (Double) Inequality}
			The simultaneous inequality $a < x \leq b$ is equivalent to:
			\[a < x \wedge x \leq b\]
			\[x \in (a, b]\]
		\subsection{Inequations (I)}
			The inequation
			\[|E(x)| < a \quad | \quad a \geq 0\]
			is equivalent to:
			\[-a < E(x) < a \qquad \text{or} \qquad -a < E(x) \wedge E(x) < a\]

			Never write $E(x) < \pm a$ or you will go deaf.
		\subsection{Inequations (II)}
			The inequation
			\[|E(x)| > a \quad | \quad a \geq 0\]
			is equivalent to:
			\[E(x) < -a \vee E(x) > a\]
	\section{Solving Polynomial Inequalities}
		\subsection{Sign Chart Method}
			Use a sign chart to specify the sign of each factor and then combine them to find the sign of the whole factored polynomial.
		\subsection{Graphical Method}
			Graph the factored polynomial and then conclude about its sign.
		\subsection{Algorithm to Solve Polynomial Inequalities}
			In order to solve an inequality involving a polynomial expression:
			\begin{itemize}
				\item Move all terms to one side of the inequality
				\item Factor the polynomial
				\item Use the sign chart or graphical method to find the sign of the polynomial
				\item Write the solution set
			\end{itemize}
		\subsection{Technology}
			When the polynomial is not factored, use technology to find the solution.
