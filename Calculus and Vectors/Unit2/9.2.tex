\section{Intersection of a Line with a Plane}
\subsection{Relative Position of a Line and a Plane}
	There are three possible situations:
	\begin{enumerate}
		\item The line \emph{intersects} the plane at a single point. \[P = L \cap \pi\]
		\item The line \emph{lies} on the plane. There are an infinite number of points of intersection. \[L = L \cap \pi\]
		\item The line is \emph{parallel} to the plane but \emph{distinct}. There is no point of intersection. \[L \cap \pi = \varnothing\]
	\end{enumerate}
\subsection{Intersection of a Line and a Plane (Algebraic Method)}
	To get the intersection between a line $L$ and a plane $\pi$:
	\begin{enumerate}
		\item \emph{Substitute} the parametric equations of the line
			\begin{equation*}
				L:
				\begin{cases}
					x = x_0 + tu_x\\
					y = y_0 + tu_y\\
					z = z_0 + tu_z
				\end{cases}
				t \in \real
			\end{equation*}
			into the Cartesian equation of the plane
			\[\pi: Ax + By + Cz + D = 0\]
			to get the equation:
			\begin{equation}\label{eq:9.2:combined}
				A(x_0 + tu_x) + B(y_0 + tu_y) + C(z_0 + tu_z) + D = 0
			\end{equation}
		\item \emph{Solve} (if possible) the equation \eqref{eq:9.2:combined} for the parameter $t$.
		\item \emph{Substitute} the value of the parameter $t$ into the parametric equations of the line to get the point of intersection.
	\end{enumerate}
\subsection{Unique Solution (Point Intersection)}
	In this case, by solving the equation you get a \emph{unique value} for the parameter $t$.
	Therefore, there is a unique \emph{point of intersection} between the line and the plane.
	\[P = L \cap \pi\]
	The line \emph{intersects} the plane at a unique point.
\subsection{Infinite Number of Solutions (Line Intersection)}
	In this case, by solving the equation \eqref{eq:9.2:combined} you get the equation:
	\[0t = 0\]
	which has an \emph{infinite number of solutions}.
	Therefore, there are an \emph{infinite number of points of intersection}.
	\[L = L \cap \pi\]
	The line \emph{lies} on the plane.
\subsection{No Solution (No Intersection)}
	In this case, by solving the equation \eqref{eq:9.2:combined} you get a false statement like:
	\[0t = 1\]
	The equation \emph{does not have any solution} and therefore there is \emph{no point of intersection} between the line and the plane.
	\[L \cap \pi = \varnothing\]
	The line is \emph{parallel} to the plane and \emph{does not lie} on the plane.
\subsection{Classifying Lines}
	Consider the line $L: \vv{r} = \vv{r_0} + t\vv{u} \such t \in \real$, where $P_0(x_0,y_0,z_0)$ is a specific point on the line, and the plane $\pi: Ax + By + Cz + D = 0$, where $\vv{n} = (A, B, C)$ is a normal vector to the plane.
	\begin{enumerate}
		\item If $\vv{n} \cdot \vv{u} \neq 0$ the line \emph{intersects} the plane at a unique point. \[P = L \cap \pi\]
		\item If $\vv{n} \cdot \vv{u} = 0$ and $Ax_0 + By_0 + Cz_0 + D = 0$ then the line \emph{lies} on the plane. \[L = L \cap \pi\]
		\item If $\vv{n} \cdot \vv{u} = 0$ and $Ax_0 + By_0 + Cz_0 + D \neq 0$ then the line is \emph{parallel} to the plane but \emph{does not lie} on the plane. \[L \cap \pi = \varnothing\]
	\end{enumerate}
	\textbf{Note. By solving the equation \eqref{eq:9.2:combined} for $t$ you will end by getting the same cases and conditions as above.}
