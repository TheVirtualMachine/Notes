\section{Intersection of Two Lines} \subsection{Relative Position of Two Lines}
	Two lines may be:
	\begin{enumerate}
		\item Parallel and distinct.
		\item Parallel and coincident.
		\item Intersecting (not parallel).
		\item Skew (not parallel, not intersecting).
	\end{enumerate}
\subsection{Intersection of Two Lines (Algebraic Method)}
	The point of intersection of two lines $L_1: \vv{r} = \vv{r_{01}} + t\vv{u_1} \such t \in \real$ and $L_2: \vv{r} = \vv{r_{02}} + s\vv{u_2} \such s \in \real$ is given by the \emph{solution} of the following system of equations (if it exists):
	\begin{equation*}
		\begin{cases}
			x_{01} + tu_{x1} = x_{02} + su_{x2}\\
			y_{01} + tu_{y1} = y_{02} + su_{y2}\\
			z_{01} + tu_{z1} = z_{02} + su_{z2}
		\end{cases}
		s, t \in \real
	\end{equation*}
	\textbf{Hint: Solve by \emph{substitution} or \emph{elimination} the system of two equations and \emph{check} if the third is satisfied.}
\subsection{Unique Solution}
	If by solving the system you end by getting a \emph{unique} value for $t$ and $s$ satisfying this system, then the lines have a \emph{unique point of intersection}.
	To get this point, substitute either the $t$ value into the line $L_1$ equation or substitute the $s$ value into the line $L_2$ equation.
\subsection{Infinite Number of Solutions}
	If by solving the system you end by getting two true statements (like $2=2$) and one equation in $s$ and $t$, then there exist an \emph{infinite number of solutions} of the system.
	Therefore the lines intersect at an \emph{infinite number of points}. 
	In this case the lines are parallel and coincident.
\subsection{No Solution (Parallel Lines)}
	If by solving the system you get at least one \emph{false} statement (like $0=1$) then the system has \emph{no solution}.
	Therefore, the lines have \emph{no point of intersection}.
	If, in addition, the lines are parallel ($\vv{u_1} \times \vv{u_2} = \vv{0}$), then the lines are \emph{parallel and distinct}.
\subsection{No Solution (Skew Lines)}
	If by solving the system you get at least one \emph{false} statement (like $0=1$) then the system has \emph{no solution}.
	Therefore, the lines have \emph{no point of intersection}.
	If, in addition, the lines are \emph{not parallel} ($\vv{u_1} \times \vv{u_2} \neq \vv{0}$), then the lines are \emph{skew}.
\subsection{Classifying Lines (Vector Method)}
	\begin{center}
		\begin{forest}
			for tree={s sep=3cm}
			[{Parallel lines\\$\vv{u_1} \times \vv{u_2} = \vv{0}$}, align=center
				[Parallel coincident lines, edge label={node[midway,above left]{$(\vv{r_{01}}-\vv{r_{02}}) \times \vv{u_1} = \vv{0}$}}]
				[Parallel distinct lines, edge label={node[midway,above right]{$(\vv{r_{01}}-\vv{r_{02}}) \times \vv{u_1} \neq \vv{0}$}}]
			]
		\end{forest}

		\vspace{1cm}

		\begin{forest}
			for tree={s sep=3cm}
			[Nonparallel lines\\$\vv{u_1} \times \vv{u_2} \neq \vv{0}$, align=center
				[Nonparallel intersecting lines, edge label={node[midway,above left]{$(\vv{r_{01}}-\vv{r_{02}})\cdot(\vv{u_1}\times\vv{u_2}) = 0$}}]
				[Nonparallel skew lines, edge label={node[midway,above right]{$(\vv{r_{01}}-\vv{r_{02}})\cdot(\vv{u_1}\times\vv{u_2}) \neq 0$}}]
			]
		\end{forest}
	\end{center}
