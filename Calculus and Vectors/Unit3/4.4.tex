\section{Concavity and Points of Inflection}
\subsection{Concavity Upward}
	The graph of the function $f$ has a \emph{concavity upward} if:
	\begin{itemize}
		\item Graph lies above all its tangents.
		\item Tangents rotate counter-clockwise.
		\item $m_T = IRC = f'(x)$ increases
		\item $f''(x) > 0$
	\end{itemize}
\subsection{Concavity Downward}
	The graph of the function $f$ has a \emph{concavity downward} if:
	\begin{itemize}
		\item Graph lies below all its tangents.
		\item Tangents rotate clockwise.
		\item $m_T = IRC = f'(x)$ decreases
		\item $f''(x) < 0$
	\end{itemize}
\subsection{Test for Concavity}
	Let $f$ be a function twice differentiable ($f''(x)$ exits) over $(a,b)$.
	\begin{enumerate}
		\item If $f''(x) > 0$ for all $x \in (a,b)$, then the function $f$ (or its graph) is \emph{concave upward} over $(a,b)$.
		\item If $f''(x) < 0$ for all $x \in (a,b)$, then the function $f$ (or its graph) is \emph{concave downward} over $(a,b)$.
		\item If $f''(x) = 0$ for all $x \in (a,b)$, then the function $f$ (or its graph) has \emph{no concavity} over $(a,b)$. In this case, $f'(m) = m$, and $f(x) = mx + b$. The function is linear and its graph is a straight line.
	\end{enumerate}
\subsection{Point of Inflection}
	A point $P(i,f(i))$ on the graph of $y=f(x)$ is called a point of inflection if the concavity of the graph \emph{changes} at $P$ from concave upward to concave downward or from concave downward to concave upward.
\subsection{Second Derivative Test}
	Let $f$ be a twice differentiable function over an open interval containing the critical number $c$ and $f'(c) = 0$.
	\begin{enumerate}
		\item If $f''(c) > 0$ then $f$ has a local minimum at $x=c$.
		\item If $f''(c) < 0$ then $f$ has a local maximum at $x=c$.
		\item If $f''(c) = 0$ then $f$ may have a local minimum, maximum, or neither (inconclusive case). Use the first derivative test to conclude.
	\end{enumerate}
