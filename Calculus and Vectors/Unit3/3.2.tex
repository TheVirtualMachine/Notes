\section{Maximum and Minimum on an Interval: Extreme Values}
\subsection{Global Maximum}
	A function $f$ has a \emph{global (absolute) maximum} at $x=c$ is $f(x) \leq f(c)$ for all $x \in D_f$.

	$f(c)$ is called the \emph{global (absolute) maximum value}.

	$(c, f(c))$ is called the \emph{global (absolute) maximum point}.

	\textbf{Note. An \emph{extremum} is either a minimum or maximum (value, point, local, or global).}
\subsection{Global Minimum}
	A function $f$ has a \emph{global (absolute) minimum} at $x = c$ is $f(x) \geq f(c)$ for all $x \in D_f$.

	$f(c)$ is called the \emph{global (absolute) minimum value}.

	$(c, f(c))$ is called the \emph{global (absolute) minimum point}.

	\textbf{Notes:}
	\begin{description}
		\item[Extrema] The plural of extremum.
		\item[Minima] The plural of minimum.
		\item[Maxima] The plural of maximum.
	\end{description}
\subsection{Global (Absolute) Extrema Algorithm}
	To find the global (absolute) extrema for a \emph{continuous} function $f$ over a close interval $[a,b]$:
	\begin{enumerate}
		\item Identify all \emph{critical} numbers over $(a,b)$.
		\item Find the \emph{values} of the function $f(c)$ at each critical number $c$ in $(a,b)$.
		\item Find the \emph{values} $f(a)$ and $f(b)$.
		\item From the values obtained at part 2 and 3:
			\begin{itemize}
				\item The \emph{largest} represents the \emph{global (absolute) maximum} value.
				\item The \emph{smallest} represents the \emph{global (absolute) minimum} value.
			\end{itemize}
	\end{enumerate}
	\textbf{Note. A \emph{critical number} $c$ is a number such that $f'(c) = 0$ or $f'(c) = \mathrm{DNE}$.}
