\section{Asymptotes}
\subsection{Vertical Asymptote}
	If the value of $f(x)$ can be made \emph{arbitrarily large} by taking $x$ \emph{sufficiently close} to $a$ with $x < a$ then $\displaystyle \lim_{x \to a^-} f(x) = \infty$. The line $x=a$ is called a \emph{vertical asymptote} to the graph of $y=f(x)$.

	\textbf{Notes:}
	\begin{enumerate}
		\item A function of the form $f(x) = \frac{p(x)}{q(x)}$ has a vertical asymptote at $x=a$ if $p(a) \neq 0 \land q(a) = 0$.
		\item A function of the form $f(x) = p(x)\log_b q(x)$ has a vertical asymptote $x=a$ if $p(a) \neq 0 \land q(a) = 0$.
	\end{enumerate}
\subsection{Horizontal Asymptote}
	A horizontal line $y=b$ is called a horizontal asymptote to the graph of $y=f(x)$ if $\displaystyle \lim_{x \to \pm\infty}f(x) = b$.

	\textbf{Notes:}
	\begin{enumerate}
		\item A horizontal asymptote may be crossed or touched by the graph of the function.
		\item The graph of a function may have at most two horizontal asymptotes (one as $x \to -\infty$ and one as $x \to +\infty$).
	\end{enumerate}
\subsection{Limits at Infinity}
	If $a > 0$, then:
	\[\lim_{x \to \pm\infty} x^a = (\pm\infty)^a\]
	\[\lim_{x \to \pm\infty} \frac{1}{x^a} = \frac{1}{(\pm\infty)^a} = 0\]
	\[\lim_{x \to \pm\infty} (a_nx^n + a_{n-1}x^{n-1} + \dots + a_2x^2 + a_1x + a_0) = \lim_{x \to \pm\infty} a_nx^n\]
	\[\lim_{x \to \pm\infty} \frac{a_nx^n + a_{n-1}x^{n-1} + \dots + a_2x^2 + a_1x + a_0}{b_mx^m + b_{m-1}x^{m-1} + \dots + b_2x^2 + b_1x + b_0} = \lim_{x \to \pm\infty} \frac{a_nx^n}{b_mx^m}\]
\subsection{Horizontal Asymptotes for Rational Functions}
	A rational function of the form:
	\[f(x) = \frac{P_n(x)}{Q_m(x)} = \frac{a_nx^n + a_{n-1}x^{n-1} + \dots + a_2x^2 + a_1x + a_0}{b_mx^m + b_{m-1}x^{m-1} + \dots + b_2x^2 + b_1x + b_0}\]
	has:
	\begin{itemize}
		\item A horizontal asymptote $y=0$ if $m>n$.
		\item A horizontal asymptote $y=\frac{a_n}{m_n}$ if $m=n$.
		\item No horizontal asymptote if $m<n$.
	\end{itemize}
	\textbf{Note. A rational function may have at most one horizontal asymptote.}
\subsection{Oblique (Slant) Asymptote}
	The line $y=ax + b$ is an oblique (slant) asymptote for the curve $y=f(x)$ if:
	\[\lim_{x \to \pm\infty} \left( f(x) - (ax+b) \right) = 0\]
	\textbf{Notes:}
	\begin{enumerate}
		\item An oblique asymptote may be crossed or touched by the graph of the function.
		\item The graph of a function may have at most two oblique asymptotes (one as $x \to -\infty$ and one as $x \to +\infty$).
		\item The graph of a function may have one horizontal asymptote and one oblique asymptote (one as $x \to -\infty$ and the other as $x \to +\infty$).
	\end{enumerate}
\subsection{Oblique Asymptotes for Rational Functions}
	A rational function of the form:
	\[f(x) = \frac{P_n(x)}{Q_m(x)} = \frac{a_nx^n + a_{n-1}x^{n-1} + \dots + a_2x^2 + a_1x + a_0}{b_mx^m + b_{m-1}x^{m-1} + \dots + b_2x^2 + b_1x + b_0}\]
	has an oblique (slant) asymptote if $n=m+1$.

	\textbf{Note. To get the equation of the oblique (slant) asymptote, use the \emph{long division algorithm} to write the rational function in the form:}
	\[f(x) = \frac{P_n(x)}{Q_m(x)} = ax+b+\frac{R(x)}{Q_m(x)}\]
	\textbf{where $0 \leq \operatorname{degree}(R) < \operatorname{degree}(Q_m)$.}
\subsection{Oblique Asymptotes for any Functions}
	The oblique asymptote $y=ax+b$ may be obtained by computing the following two limits at infinity:
	\[a = \lim_{x \to \pm\infty} \frac{f(x)}{x}\]
	\[b = \lim_{x \to \pm\infty} \left( f(x) - ax \right)\]
	\textbf{Note. In order for the oblique asymptote to be defined, both limits above must exist (must be finite numbers).}
