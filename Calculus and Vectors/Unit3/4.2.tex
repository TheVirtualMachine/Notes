\section{The Mean Value Theorem (AP)}
\subsection{Rolle's Theorem}
	Let $y=f(x)$ be a function continuous on $[a,b]$ and differentiable on $(a,b)$.

	If $f(a) = f(b)$ then there is a number $c \in (a,b)$ such that $f'(c) = 0$.

	\textbf{Note. Tangent line is horizontal at $P(c, f(c))$.}
\subsection{Mean Value Theorem}
	Let $y=f(x)$ be a function continuous on $[a,b]$ and differentiable on $(a,b)$.
	Then there is a number $c \in (a,b)$ such that $f'(c) = \frac{f(b) - f(a)}{b-a}$.

	\textbf{Note. Slope of tangent line at $P(c,f(c))$ is equal to slope of secant line.}
\subsection{Theorem}
	If $f'(x) = 0$ for all $x \in (a,b)$, then $f$ is constant on $(a,b)$.
\subsection{Theorem}
	If $f'(x) = g'(x)$ for all $x \in (a,b)$, then $f-g$ is constant on $(a,b)$ and $f(x) = g(x) + c$ where $c$ is a constant.
