\section{Critical Points: Local Maxima and Minima}
\setcounter{subsection}{1}
\subsection{Local Maximum}
	A function $f$ has a \emph{local (relative) maximum} at $x=c$ if:
	\begin{itemize}
		\item $f(x) \leq f(c)$ when $x$ is sufficiently close to $c$ (from both sides).
		\item $f'(x)$ changes sign from positive to negative at $c$.
	\end{itemize}
\subsection{Local Minimum}
	A function $f$ has a \emph{local (relative) minimum} at $x=c$ if::
	\begin{itemize}
		\item $f(x) \geq f(c)$ when $x$ is sufficiently close to $c$ (from both sides).
		\item $f'(x)$ changes sign from negative to positive at $c$.
	\end{itemize}
\subsection{Critical Numbers and Critical Points}
	The number $c \in D_f$ is a \emph{critical number} if:
	\[f'(c) = 0 \quad \text{or} \quad f'(c)\ \mathrm{DNE}\]
	The point $P(c,f(c))$ is called a \emph{critical point}.
	\textbf{Notes:}
	\begin{enumerate}
		\item A local extremum happens always at a critical point (Fermat's theorem).
		\item At a critical number a function may or may not have a local extremum.
	\end{enumerate}
