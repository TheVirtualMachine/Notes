\section{Vector and Parametric Equations of a Line in $\real^2$}
\subsection{Vector Equation of a Line in $\real^2$}
	Consider the line $L$ that passes through the point $P_0 (x_0, y_0)$ and is parallel to the vector $\vv{u}$.
	The point $P(x,y)$ is a \emph{generic point} on the line.

	\[\vv{P_0P} = t\vv{u}\]
	\[\vv{OP} - \vv{OP_0} = t\vv{u}\]
	\[\vv{r} - \vv{r_0} = t\vv{u}\]

	The \emph{vector equation} of the line is:
	\[\vv{r} = \vv{r_0} + t\vv{u} \such t \in \real\]
	Where:
	\begin{itemize}
		\item $\vv{r} = \vv{OP}$ is the \emph{position vector} of a \emph{generic point} $P$ on the line.
		\item $\vv{r_0} = \vv{OP_0}$ is the \emph{position vector} of a \emph{specific point} $P_0$ on the line.
		\item $\vv{u}$ is a vector parallel to the line called the \emph{direction vector} of the line.
		\item $t$ is a \emph{real number} corresponding to the generic point $P$.
	\end{itemize}
	\textbf{Note: The vector equation of a line is \emph{not unique}.
	It depends on the specific point $P_0$ and on the direction vector $\vv{u}$ that are used.}
\subsection{Parametric Equations of a Line in $\real^2$}
	We can rewrite the vector equation of a line:
	\[\vv{r} = \vv{r_0} + t\vv{u} \such t \in \real\]
	as:
	\[(x, y) = (x_0, y_0) + t(u_x, u_y) \such t \in \real\]
	Split this vector equation into the \emph{parametric equations} of a line in $\real^2$:
	\begin{equation*}
		\begin{cases}
			x = x_0 + tu_x\\
			y = y_0 + yu_y
		\end{cases}
		t \in \real
	\end{equation*}
\subsection{Parallel Lines}
	Two lines $L_1$ and $L_2$ with direction vectors $\vv{u_1}$ and $\vv{u_2}$ are \emph{parallel} ($L_1 \parallel L_2$) if:
	\[\vv{u_1} \parallel \vv{u_2}\]
	or, there exists $k \in \real$ such that:
	\[\vv{u_2} = k\vv{u_1}\]
	or:
	\[\vv{u_1} \times \vv{u_2} = \vv{0}\]
	or scalar components are \emph{proportional}:
	\[\frac{u_{2x}}{u_{1x}} = \frac{u_{2u}}{u_{1u}} = k\]
\subsection{Perpendicular Lines}
	Two lines $L_1$ and $L_2$ with direction vectors $\vv{u_1}$ and $\vv{u_2}$ are \emph{perpendicular} ($L_1 \perp L_2$) if:
	\[\vv{u_1} \perp \vv{u_2}\]
	or:
	\[\vv{u_1} \cdot \vv{u_2} = 0\]
	or:
	\[u_{1x}u_{2x} + u_{1y}u_{2y} = 0\]
\subsection{2D Perpendicular Vectors}
	Given a 2D vector $\vv{u} = (a, b)$, two 2D vectors perpendicular to $\vv{u}$ are $\vv{v} = (-b, a)$ and $\vv{w} = (b, -a)$.

	Indeed:
	\[\vv{u} \cdot \vv{v} = (a, b) \cdot (-b, a) = -ab + ab = 0 \implies \vv{u} \perp \vv{v}\]
\subsection{Special Lines}
	A line \emph{parallel} to the $x$-axis has a direction vector in the form $\vv{u} = (u_x, 0) \such u_x \neq 0$.

	A line \emph{parallel} to the $y$-axis has a direction vector in the form $\vv{u} = (0, u_y) \such u_y \neq 0$.
