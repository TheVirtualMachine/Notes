\section{Antiderivatives and Indefinite Integrals}
\subsection{Integration}
	Differentiation is the process of finding $f'(x)$ given $f(x)$. \emph{Integration} is the inverse process of differentiation. It means finding $f(x)$ given $f'(x)$.
	\[f(x) \xrightarrow{\text{Differentiation}} f'(x)\]
	\[f(x) \xleftarrow{\text{Integration}} f'(x)\]
\subsection{Antiderivative}
	The \emph{antiderivative} of the function $f(x)$ is a function $F(x)$ such that $F'(x) = f(x)$.
\subsection{Families of Antiderivatives}
	If $F(x)$ is an antiderivative of $f(x)$, so is $F(x) + C$, where $C$ is a constant called the \emph{constant of integration}.
\subsection{Initial Condition}
	An antiderivative of a function may be uniquely identified by an initial condition:
	\[F(a) = b\]
\subsection{Indefinite Integrals}
	The \emph{indefinite integral} is the most commonly used notation for antiderivatives.
	If $F(x)$ is an antiderivative of $f(x)$, then we write:
	\[F(x) = \int f(x) \dif x\]
	By definition:
	\[\od{}{x} \int f(x) \dif x = f(x)\]
\subsection{List of Indefinite Integrals}\label{subsec:4.9:integrals}
	\begin{align*}
		&\int x^n \dif x = \frac{x^{n+1}}{n+1} + C \such n \neq -1\\
		&\int \frac{1}{x} \dif x = \ln \abs{x} + C\\
		&\int e^x \dif x = e^x + C\\
		&\int a^x \dif x = \frac{a^x}{\ln a} + C\\
		&\int \dif x = x + C\\
		&\int \sin(x) \dif x = -\cos(x) + C\\
		&\int \cos(x) \dif x = \sin(x) + C\\
		&\int \sec^2(x) \dif x = \tan(x) + C\\
		&\int \csc^2(x) \dif x = -\cot(x) + C\\
		&\int \sec(x) \tan(x) \dif x = \sec(x) + C\\
		&\int \csc(x) \cot(x) \dif x = -\csc(x) + C\\
		&\int \frac{1}{x^2 + 1} \dif x = \arctan(x) + C\\
		&\int \frac{1}{\sqrt{1 - x^2}} \dif x = \arcsin(x) + C
	\end{align*}
\subsection{Properties of Antiderivatives or Indefinite Integrals}
	\[\int c f(x) \dif x = c \int f(x) \dif x\]
	\[\int \left( f(x) \pm g(x) \right) \dif x = \int f(x) \dif x \pm \int g(x) \dif x\]
