\section{Continuity}
\subsection{Continuity}
	A function $y=f(x)$ is continuous at a number $x=a$ if \[L = R = f(a)\]
	where:

	$\displaystyle L = \lim_{x \to a^-} f(x)$ is the left-hand limit at $x=a$.

	$\displaystyle R = \lim_{x \to a^+} f(x)$ is the right-hand limit at $x=a$.

	$f(a)$ is the value of the function at $x=a$.
	
	\textbf{Note: A function is continuous if it can be drawn without lifting your pencil from the paper.}
\subsection{Discontinuity}
	If $y=f(x)$ is not continuous at $x=a$ then we say:
	``$y=f(x)$ is discontinuous at $x=a$''
	or
	``$y=f(x)$ has a discontinuity at $x=a$''.
\subsection{Removable Discontinuity}
	A function $y=f(x)$ has a removable discontinuity at $x=a$ if:
	\begin{enumerate}
		\item $\displaystyle L=R=\lim_{x \to a} f(x)$ exists
		\item $f(a)$ DNE or $\displaystyle \lim_{x \to a}f(x) \neq f(a)$
	\end{enumerate}
	\textbf{Note: A removable discontinuity can be removed be redefining the function $x=a$ as $\displaystyle f(a) \stackrel{def}{=} \lim_{x \to a} f(x)$.}
\subsection{Jump Discontinuity}
	A function $y=f(x)$ has a jump discontinuity at $x=a$ if:
	\[L = \lim_{x \to a^-} f(x) \neq \lim_{x \to a^+} f(x) = R\]
\subsection{Infinite Discontinuity}
	A function $y=f(x)$ has an infinite discontinuity at $x=a$ if at least one side of the limit is unbounded (approaches $\infty$ or $-\infty$).
\subsection{Continuity over an Interval (AP only)}
	A function $y=f(x)$ is continuous over an open interval $(a,b)$ if the function is continuous at every number in that interval.

	A function is continuous from the right at $x=a$ if $R=f(a)$.

	A function is continuous from the left at $x=a$ if $L=f(a)$.
\subsection{Elementary Functions (AP only)}
	Elementary functions (polynomial, power, rational, trigonometric, exponential, and logarithmic) are continuous over their domain.
\subsection{Composition of Functions}
	If $g$ is continuous at $x=a$ and $f$ is continuous at $g(a)$ then $f(g(x))$ is continuous at $x=a$.
\subsection{Intermediate Value Theorem (AP only)}
	If $y=f(x)$ is a continuous function over the interval $[a,b]$ with $f(a) \neq f(b)$, then for any number $N$ between $f(a)$ and $f(b)$ there exist a number $c \in (a,b)$ such that $f(c) = N$.
